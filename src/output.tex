\documentclass{article}
\begin{document}
\section{Differentiator}
wazzzuuuup, shut up and take my derivative of this function: \newline $$ f(x) = { {  {  {  x  }  ^  {  2  }  }  +  {  {  x  }  \cdot  {  3  }  }  }  +  {  2  } } $$obviously, the derivative of multiplication looks like this: $$ ({ {  x  }  \cdot  {  3  } })' = { {  {  1  }  \cdot  {  3  }  }  +  {  {  x  }  \cdot  {  0  }  } } $$
Znamenskaya forbade doing this, but: $$ ({ {  x  }  ^  {  2  } })' = { {  {  2  }  \cdot  {  {  x  }  ^  {  {  2  }  -  {  1  }  }  }  }  \cdot  {  1  } } $$
the derivative of the sum can be represented as follows: $$ ({ {  {  x  }  ^  {  2  }  }  +  {  {  x  }  \cdot  {  3  }  } })' = { {  {  {  2  }  \cdot  {  {  x  }  ^  {  {  2  }  -  {  1  }  }  }  }  \cdot  {  1  }  }  +  {  {  {  1  }  \cdot  {  3  }  }  +  {  {  x  }  \cdot  {  0  }  }  } } $$
the derivative of the sum can be represented as follows: $$ ({ {  {  {  x  }  ^  {  2  }  }  +  {  {  x  }  \cdot  {  3  }  }  }  +  {  2  } })' = { {  {  {  {  2  }  \cdot  {  {  x  }  ^  {  {  2  }  -  {  1  }  }  }  }  \cdot  {  1  }  }  +  {  {  {  1  }  \cdot  {  3  }  }  +  {  {  x  }  \cdot  {  0  }  }  }  }  +  {  0  } } $$
with typical operations simplification$$ ({ {  {  {  x  }  ^  {  2  }  }  +  {  {  x  }  \cdot  {  3  }  }  }  +  {  2  } })' = { {  {  2  }  \cdot  {  {  x  }  ^  {  {  2  }  -  {  1  }  }  }  }  +  {  3  } } $$
with all simplification$$ ({ {  {  {  x  }  ^  {  2  }  }  +  {  {  x  }  \cdot  {  3  }  }  }  +  {  2  } })' = { {  {  2  }  \cdot  {  {  x  }  ^  {  1  }  }  }  +  {  3  } } $$
\end{document}
