\documentclass{article}
\begin{document}
\section{Differentiator}
wazzzuuuup, shut up and take my money.\newline calculate the derivative of the fraction.\newline Ostap once said: $$ ({cos  {  x  } })' = { {  { sin  {  x  }  }  \cdot  {  (  -1  )  }  }  \cdot  {  1  } } $$
every kindergartener in the USSR knew that: $$ ({sin  {  x  } })' = { { cos  {  x  }  }  \cdot  {  1  } } $$
the derivative of the sum can be represented as follows: $$ ({ { sin  {  x  }  }  +  { cos  {  x  }  } })' = { {  { cos  {  x  }  }  \cdot  {  1  }  }  +  {  {  { sin  {  x  }  }  \cdot  {  (  -1  )  }  }  \cdot  {  1  }  } } $$
the derivative of the numerator is calculated as follows: ... wait wait its the same, see above bro\newline \newline Znamenskaya forbade doing this, but: $$ ({ {  x  }  ^  {  3  } })' = { {  {  3  }  \cdot  {  {  x  }  ^  {  {  3  }  -  {  1  }  }  }  }  \cdot  {  1  } } $$
Ostap once said: $$ ({cos  {  {  x  }  ^  {  3  }  } })' = { {  { sin  {  {  x  }  ^  {  3  }  }  }  \cdot  {  (  -1  )  }  }  \cdot  {  {  {  3  }  \cdot  {  {  x  }  ^  {  {  3  }  -  {  1  }  }  }  }  \cdot  {  1  }  } } $$
Karzhemanov said that the denominator is equal to: ... wait wait its the same, see above bro\newline \newline final differentiated fraction: $$ ({ {  { sin  {  x  }  }  +  { cos  {  x  }  }  }  \over  { cos  {  {  x  }  ^  {  3  }  }  } })' = { {  {  {  (  {  { cos  {  x  }  }  \cdot  {  1  }  }  +  {  {  { sin  {  x  }  }  \cdot  {  (  -1  )  }  }  \cdot  {  1  }  }  )  }  \cdot  { cos  {  {  x  }  ^  {  3  }  }  }  }  -  {  {  (  { sin  {  x  }  }  +  { cos  {  x  }  }  )  }  \cdot  {  {  { sin  {  {  x  }  ^  {  3  }  }  }  \cdot  {  (  -1  )  }  }  \cdot  {  {  {  3  }  \cdot  {  {  x  }  ^  {  {  3  }  -  {  1  }  }  }  }  \cdot  {  1  }  }  }  }  }  \over  {  { cos  {  {  x  }  ^  {  3  }  }  }  \cdot  { cos  {  {  x  }  ^  {  3  }  }  }  } } $$
\end{document}
